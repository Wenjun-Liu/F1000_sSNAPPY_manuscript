\documentclass[9pt,a4paper,]{extarticle}

\usepackage{f1000_styles}

\usepackage[pdfborder={0 0 0}]{hyperref}

\usepackage[numbers]{natbib}
\bibliographystyle{unsrtnat}


%% maxwidth is the original width if it is less than linewidth
%% otherwise use linewidth (to make sure the graphics do not exceed the margin)
\makeatletter
\def\maxwidth{ %
  \ifdim\Gin@nat@width>\linewidth
    \linewidth
  \else
    \Gin@nat@width
  \fi
}
\makeatother

\usepackage{color}
\usepackage{fancyvrb}
\newcommand{\VerbBar}{|}
\newcommand{\VERB}{\Verb[commandchars=\\\{\}]}
\DefineVerbatimEnvironment{Highlighting}{Verbatim}{commandchars=\\\{\}}
% Add ',fontsize=\small' for more characters per line
\usepackage{framed}
\definecolor{shadecolor}{RGB}{248,248,248}
\newenvironment{Shaded}{\begin{snugshade}}{\end{snugshade}}
\newcommand{\AlertTok}[1]{\textcolor[rgb]{0.94,0.16,0.16}{#1}}
\newcommand{\AnnotationTok}[1]{\textcolor[rgb]{0.56,0.35,0.01}{\textbf{\textit{#1}}}}
\newcommand{\AttributeTok}[1]{\textcolor[rgb]{0.77,0.63,0.00}{#1}}
\newcommand{\BaseNTok}[1]{\textcolor[rgb]{0.00,0.00,0.81}{#1}}
\newcommand{\BuiltInTok}[1]{#1}
\newcommand{\CharTok}[1]{\textcolor[rgb]{0.31,0.60,0.02}{#1}}
\newcommand{\CommentTok}[1]{\textcolor[rgb]{0.56,0.35,0.01}{\textit{#1}}}
\newcommand{\CommentVarTok}[1]{\textcolor[rgb]{0.56,0.35,0.01}{\textbf{\textit{#1}}}}
\newcommand{\ConstantTok}[1]{\textcolor[rgb]{0.00,0.00,0.00}{#1}}
\newcommand{\ControlFlowTok}[1]{\textcolor[rgb]{0.13,0.29,0.53}{\textbf{#1}}}
\newcommand{\DataTypeTok}[1]{\textcolor[rgb]{0.13,0.29,0.53}{#1}}
\newcommand{\DecValTok}[1]{\textcolor[rgb]{0.00,0.00,0.81}{#1}}
\newcommand{\DocumentationTok}[1]{\textcolor[rgb]{0.56,0.35,0.01}{\textbf{\textit{#1}}}}
\newcommand{\ErrorTok}[1]{\textcolor[rgb]{0.64,0.00,0.00}{\textbf{#1}}}
\newcommand{\ExtensionTok}[1]{#1}
\newcommand{\FloatTok}[1]{\textcolor[rgb]{0.00,0.00,0.81}{#1}}
\newcommand{\FunctionTok}[1]{\textcolor[rgb]{0.00,0.00,0.00}{#1}}
\newcommand{\ImportTok}[1]{#1}
\newcommand{\InformationTok}[1]{\textcolor[rgb]{0.56,0.35,0.01}{\textbf{\textit{#1}}}}
\newcommand{\KeywordTok}[1]{\textcolor[rgb]{0.13,0.29,0.53}{\textbf{#1}}}
\newcommand{\NormalTok}[1]{#1}
\newcommand{\OperatorTok}[1]{\textcolor[rgb]{0.81,0.36,0.00}{\textbf{#1}}}
\newcommand{\OtherTok}[1]{\textcolor[rgb]{0.56,0.35,0.01}{#1}}
\newcommand{\PreprocessorTok}[1]{\textcolor[rgb]{0.56,0.35,0.01}{\textit{#1}}}
\newcommand{\RegionMarkerTok}[1]{#1}
\newcommand{\SpecialCharTok}[1]{\textcolor[rgb]{0.00,0.00,0.00}{#1}}
\newcommand{\SpecialStringTok}[1]{\textcolor[rgb]{0.31,0.60,0.02}{#1}}
\newcommand{\StringTok}[1]{\textcolor[rgb]{0.31,0.60,0.02}{#1}}
\newcommand{\VariableTok}[1]{\textcolor[rgb]{0.00,0.00,0.00}{#1}}
\newcommand{\VerbatimStringTok}[1]{\textcolor[rgb]{0.31,0.60,0.02}{#1}}
\newcommand{\WarningTok}[1]{\textcolor[rgb]{0.56,0.35,0.01}{\textbf{\textit{#1}}}}

% disable code chunks background
%\renewenvironment{Shaded}{}{}

% disable section numbers
\setcounter{secnumdepth}{0}

%% added by MLS, this is not in the F1000 style by default %%

\hypersetup{unicode=true,
            pdftitle={sSNAPPY: a R/Bioconductor package for single-sample directional pathway perturbation analysis},
            pdfkeywords={RNA-seq, pathway enrichment, R package, topology, KEGG, scRNA-seq},
            pdfborder={0 0 0},
            breaklinks=true}

%% End added by MLS %%

\setlength{\parindent}{0pt}
\setlength{\parskip}{6pt plus 2pt minus 1pt}


\usepackage{endfloat}
\usepackage{booktabs}
\usepackage{longtable}
\usepackage{booktabs}
\usepackage{longtable}
\usepackage{array}
\usepackage{multirow}
\usepackage{wrapfig}
\usepackage{float}
\usepackage{colortbl}
\usepackage{pdflscape}
\usepackage{tabu}
\usepackage{threeparttable}
\usepackage{threeparttablex}
\usepackage[normalem]{ulem}
\usepackage{makecell}
\usepackage{xcolor}

\begin{document}
\pagestyle{front}

\title{sSNAPPY: a R/Bioconductor package for single-sample directional pathway perturbation analysis}

\author[1]{Wenjun Liu\thanks{\ttfamily Corresponding Author wenjun.liu@adelaide.edu.au}}
\author[2,3,4,5]{Ville-Petteri Mäkinen}
\author[1]{Wayne D. Tilley}
\author[1,6,7]{Stephen M. Pederson}
\affil[1]{Dame Roma Mitchell Cancer Research Laboratories, Adelaide Medical School, Faculty of Health and Medical Sciences, University of Adelaide, Adelaide, Australia}
\affil[2]{Australian Centre for Precision Health, Cancer Research Institute, University of South Australia, Adelaide, Australia}
\affil[3]{Computational and Systems Biology Program, Precision Medicine Theme, South Australian Health and Medical Research Institute, Adelaide, Australia}
\affil[4]{Computational Medicine, Faculty of Medicine, University of Oulu, Oulu, Finland}
\affil[5]{Center for Life Course Health Research, Faculty of Medicine, University of Oulu, Oulu, Finland}
\affil[6]{Black Ochre Data Laboratories, Telethon Kids Institute, Adelaide, Australia}
\affil[7]{John Curtin School of Medical Research, Australian National University, Canberra, Australia}

\maketitle
\thispagestyle{front}

\begin{abstract}
When analysing RNA-Seq data, a common outcome is to detect biological pathways with significantly altered activity between the conditions under investigation. The most common strategies test for over-representation within pre-defined gene-sets for genes showing changed expression\citep{Subramanian2005-lx, Young2010-jw}, but without accounting for gene-gene interactions encoded by pathway topologies, and not being able to directly predict the directional change of pathway activity. To address these issues, we have deloped a single-sample pathway perturbation analysis method \href{https://bioconductor.org/packages/sSNAPPY}{\emph{sSNAPPY}}, now available as an R/Bioconductor package, which leverages pathway topology information to compute pathway perturbation scores, and predicts the potential direction of change across a set of pathways. Here, we demonstrate the use of \emph{sSNAPPY} by applying the method to public scRNA-seq data, derived from ovarian cancer patient tissues collected before and after chemotherapy. Not only were we able to replicate results reported in the original study, but \emph{sSNAPPY} was also able to detect significant perturbation of other biological processes, yielding far greater insight into the response to treatment. \emph{sSNAPPY} represents a novel pathway analysis strategy that takes into consideration of pathway topology to predict impacted biology pathways, both within related samples and across treatment groups. In addition to not relying on the detection of differentially expressed genes, the method and associated R package offer important flexibility and provide powerful visualisation tools.
\end{abstract}

\section*{Keywords}
RNA-seq, pathway enrichment, R package, topology, KEGG, scRNA-seq


\clearpage
\pagestyle{main}

\textbf{R version}: R version 4.2.2 (2022-10-31)

\textbf{Bioconductor version}: 3.16

\textbf{Package}: 1.3.4

\hypertarget{introduction}{%
\section{Introduction}\label{introduction}}

Using pathway enrichment analysis to gain biological insights from gene expression data is a pivotal step in the analysis and interpretation of RNA-seq data, for which numerous methods have been developed (reviewed in \citep{Maleki2020-ur, Mubeen2022-eq}).
Many existing methods tend to view pathways simply as a collection of gene names, as seen in those relying on the detection of differentially expressed genes and applying over-representation analysis (ORA) strategies, and those scoring all genes using functional class scoring (FCS), such as in Gene Set Enrichment Analysis (GSEA) \citep{Subramanian2005-lx}, arguably the most widely-used approach.
However, databases such as the Kyoto Encyclopaedia of Genes and Genomes (KEGG)\citep{OgataKEGGKyotoEncyclopediaa} and WikiPathways\citep{Martens2021} capture not only which genes are implicated in a certain biological process but also their interactions, activating or inhibitory roles, and their relative importance within the pathway, all of which are overlooked in ORA- and FCS-based approaches.

To fully utilise that additional information, the latest generation of pathway analysis approaches include many which are topology-based such as SPIA\citep{Tarca2009-nf}, DEGraph\citep{Jacob2012}, NetGSA\citep{Ma2016} and PRS\citep{Ibrahim2012}, as well as others which explicitly model inter-gene correlations\citep{Wu2012}.
Despite differences in the null hypotheses tested across these approaches, overall, they have demonstrated enhanced sensitivity and specificity due to their abilities to take gene-gene interconnections into account\citep{Nguyen2019-va, Ma2019}.
Nevertheless, most topology-based methods focus only on comparing activities of pathways between two treatment groups and cannot be used to score individual samples.
However, in heterogenous data where more than one factor may be influencing our observations\citep{Hanzelmann2013}, incorporating scoring within paired samples may be desirable and may be able to reveal more nuanced insights.
To address this gap, we present a Single-Sample Network and Pathway Perturbation analysis methodology called \href{https://bioconductor.org/packages/sSNAPPY}{\emph{sSNAPPY}}, available as an R/Bioconductor package.
This article defines how \emph{sSNAPPY} computes changes in gene expression within paired samples, and propagates this through gene-set topologies to predict the perturbation in pathway activities within paired samples, before providing summarised results across an entire dataset which would include more robust levels of biological replication.
The practical usage of the \emph{sSNAPPY} R/Bioconductor package will be illustrated through the analysis of a public scRNA-seq dataset using pseudo-bulk strategies.

\hypertarget{methods}{%
\section{Methods}\label{methods}}

\hypertarget{implementation}{%
\subsection{Implementation}\label{implementation}}

\emph{sSNAPPY} is an R package that has been reviewed and published on the open-source bioinformatics software platform \href{https://bioconductor.org/packages/release/bioc/html/*sSNAPPY*.html}{Bioconductor} with all source code available via GitHub.
The methodology itself is topology-based, designed to compute directional, single-sample, pathway perturbation scores in gene expression datasets with a matched-pair, or nested design (eg. samples collected before and after treatment).
This allows for the detection of pathway perturbations within all samples from a treatment group, but also within individual samples.

To run \emph{sSNAPPY}, the only required data is a log-transformed expression matrix (e.g.~logCPM) with matching sample metadata describing treatment groups and the nested structure.
It is assumed that all pre-processing has been performed beforehand, such as the exclusion of low-signal genes or normalisation to minimise technical artefacts like GC-bias.
The first step, performed internally by \emph{sSNAPPY}, is to estimate sample-specific log fold-change (\(\delta_{ghi} = \mu_{ghi} - \mu_{g0i}\)) for a treatment \(h\) across all genes \(g\) within each sample \(i\), by subtracting expression estimates for the baseline samples \(\hat\mu_{g0i}\) from those in the treatment group \(h\).
Since it has been shown that in RNA-seq data, genes with lower expression tend to have larger variance and larger estimates of change\citep{Law2014}, we utilise a gene-level weighting strategy to de-emphasise logFC estimates for low-abundance genes.
Gene-level weights \(w_g\) are obtained in a treatment-agnostic manner by fitting a loess curve through the relationship between observed gene-level variance (\(\widehat\sigma_g\)) and average signal (\(\bar\mu_{g\cdot\cdot}\)), and taking the inverse of the loess-predicted variance as the weight \(w_g = a / f(\bar\mu_{g\cdot\cdot})\), where \(f(\bar\mu_{g\cdot\cdot})\) is the predicted value from the loess curve and the constant \(a\) ensures \(\sum w_g = 1\).
We then use these weighted estimates of logFC for calculation of all pathway perturbation scores.

\emph{sSNAPPY} was built upon the group-level topology-based scoring algorithm initially proposed in R package SPIA\citep{Tarca2009} to propagate genes' changes in expression through pathway topologies to compute a perturbation score for each pathway.
By modifying the algorithm to incorporate single-sample, weighted estimates of changes in expression we are able to quantify changes in a pathway within a given sample, and then model these across all samples within a treatment group.
Thus, we define the single-sample perturbation score (\(S_{hip}\)) for a given pathway \(p\), sample \(i\) and treatment \(h\):

\[
\begin{aligned}
S_{hip} = \sum_{g \in G_p} \lbrack S_{ghip} - \delta_{ghi}^*\rbrack \text{, where} \\
S_{ghip} = \delta_{ghi}^* + \sum_{g' \in U_{gp}} \beta_{gg'p} \frac{S_{g'hip}}{N_{g'p}} 
\end{aligned}
\]
where:

\begin{itemize}
\item
  \(G_p\) represents the set of genes in pathway \(p\), such that \(g \in G_p\)
\item
  \(S_{ghip}\) is the gene-, treatment- and sample-specific perturbation score for pathway \(p\)
\item
  \(\delta_{ghi}^* = w_g\delta_{ghi}\) is the weighted logFC of gene \(g\) as described above
\item
  \(U_{gp}\) is the subset of \(G_p\) containing only the genes directly upstream of gene \(g\)
\item
  \(\beta_{gg'p}\) is the pair-wise gene-gene interactions\citep{Tarca2009} encoded by the topology matrix for genes \(g\) and \(g'\)
\item
  \(N_{gp}\) is the number of downstream genes from any gene \(g\)
\item
  \(S_{hip}\) is the accumulated pathway perturbation score for pathway \(p\) in treatment \(h\) for sample \(i\)
\end{itemize}

The Bioconductor package graphite\citep{Sales2012} provides functions that can be used to retrieve pathway topologies from a database and convert topology information to adjacency matrices.
In order to streamline this process we have implemented a convenience function, where users only need to provide the name of the desired database to retrieve all topology information in the format required by the scoring algorithm with the correct type of gene identifiers (ie. Entrez ID).

To scale the single-sample pathway perturbation scores (\(S_{hip}\)) so they are comparable across pathways and to test for significance of individual scores, null distributions of perturbation scores for each pathway are generated through a sample permutation strategy, retaining the correct correlation structure between genes within a pathway.
With each permutation, column names (i.e.~sample labels) for the logCPM matrix are randomly shuffled while the rest of the scoring algorithm remains unchanged.
We recommend users to perform a minimum of 1000 permutations, requiring at least 8 unique samples.
Subsequently, the median and median absolute deviation (MAD) of the permuted perturbation scores will be calculated and used to normalise the raw perturbation scores to robust \(Z\)-scores and obtain associated two-sided \emph{p}-values.
Since the method is single sample-based, the permutation strategy remains applicable regardless of experimental design.

Apart from assessing whether a pathway's activity changed significantly within an individual sample, users may also be interested in detecting changes at the group-level, which can be performed by modelling scores with regression models, incorporating Smyth's moderated \(t\)-statistics\citep{Smyth_2004} as implemented in \emph{limma}\citep{limma_2015}.
The single-sample nature of \emph{sSNAPPY}'s pathway perturbation scores is particularly helpful for datasets with complex experimental designs or known confounding factors as these can also be incorporated into the final regression models.

\hypertarget{operation}{%
\subsection{Operation}\label{operation}}

The package has been tested on all operating systems, requiring R \textgreater{} 4.2.0, and can be installed using BiocManager as follows.

\begin{Shaded}
\begin{Highlighting}[]
\ControlFlowTok{if}\NormalTok{ (}\SpecialCharTok{!}\FunctionTok{requireNamespace}\NormalTok{(}\StringTok{"BiocManager"}\NormalTok{, }\AttributeTok{quietly =} \ConstantTok{TRUE}\NormalTok{))}
  \FunctionTok{install.packages}\NormalTok{(}\StringTok{"BiocManager"}\NormalTok{)}
\NormalTok{BiocManager}\SpecialCharTok{::}\FunctionTok{install}\NormalTok{(}\StringTok{"sSNAPPY"}\NormalTok{)}
\end{Highlighting}
\end{Shaded}

\hypertarget{use-cases}{%
\section{Use Cases}\label{use-cases}}

\hypertarget{data}{%
\subsection{Data}\label{data}}

To demonstrate the application of \emph{sSNAPPY}, we used pre-processed counts from a publicly available scRNA-seq dataset, retrieved from Gene Expression Omnibus (GEO) with accession code GSE165897.
This dataset consists of 11 homogeneously treated high-grade serous ovarian cancer (HGSOC) patients, with samples taken before treatment and after chemotherapy\citep{Zhang2022}.
In the original study, cells were classified into epithelial, stromal, and immune cells, and we have chosen to only focuses on epithelial cells as they were what the original study primarily focused on.
Since \emph{sSNAPPY} was designed for bulk RNA-seq data, counts of epithelial cells from the same samples were first summed into pseudo-bulk profiles, giving rise to a total of 22 samples.
We considered a gene detectable if we observed \textgreater1.5 counts per million in \textgreater11 samples out of 22, representing all samples from a complete treatment group.
11,101 (33.8\%) of the 32,847 annotated genes passed this selection criteria and were included for downstream analyses.
Conditional quantile normalisation\citep{Hansen2012} was then applied to mitigate potential biases introduced by gene length and GC content.
The normalised logCPM matrix of the processed dataset and sample metadata can be downloaded from \href{https://github.com/Wenjun-Liu/F1000_sSNAPPY_manuscript/tree/master/data}{here}.

The following packages are required for this workflow

\begin{Shaded}
\begin{Highlighting}[]
\FunctionTok{library}\NormalTok{(sSNAPPY)}
\FunctionTok{library}\NormalTok{(tidyverse)}
\FunctionTok{library}\NormalTok{(magrittr)}
\FunctionTok{library}\NormalTok{(ggplot2)}
\FunctionTok{library}\NormalTok{(cowplot)}
\FunctionTok{library}\NormalTok{(kableExtra)}
\FunctionTok{library}\NormalTok{(AnnotationHub) }
\FunctionTok{library}\NormalTok{(edgeR)}
\end{Highlighting}
\end{Shaded}

To read in the data:

\begin{Shaded}
\begin{Highlighting}[]
\NormalTok{logCPM }\OtherTok{\textless{}{-}} \FunctionTok{read\_tsv}\NormalTok{(here}\SpecialCharTok{::}\FunctionTok{here}\NormalTok{(}\StringTok{"data/logCPM.tsv"}\NormalTok{)) }\SpecialCharTok{\%\textgreater{}\%}
    \FunctionTok{column\_to\_rownames}\NormalTok{(}\StringTok{"entrezid"}\NormalTok{)}
\NormalTok{sample\_meta }\OtherTok{\textless{}{-}} \FunctionTok{read\_tsv}\NormalTok{(here}\SpecialCharTok{::}\FunctionTok{here}\NormalTok{(}\StringTok{"data/sample\_meta.tsv"}\NormalTok{))}
\FunctionTok{head}\NormalTok{(sample\_meta)}
\end{Highlighting}
\end{Shaded}

\begin{verbatim}
## # A tibble: 6 x 8
##   sample        treatment patient_id anatomical_location   Age Stage   PFI   CRS
##   <chr>         <chr>     <chr>      <chr>               <dbl> <chr> <dbl> <dbl>
## 1 EOC372_treat~ treatmen~ EOC372     Peritoneum             68 IIIC    460     1
## 2 EOC372_post-~ post-NACT EOC372     Peritoneum             68 IIIC    460     1
## 3 EOC443_post-~ post-NACT EOC443     Omentum                54 IVA     177     3
## 4 EOC443_treat~ treatmen~ EOC443     Omentum                54 IVA     177     3
## 5 EOC540_treat~ treatmen~ EOC540     Omentum                62 IIIC    126     2
## 6 EOC540_post-~ post-NACT EOC540     Omentum                62 IIIC    126     2
\end{verbatim}

Note that to apply sSNAPPY, the rownames of the logCPM matrix must be in Entrez IDs.

\hypertarget{retrieval-pathway-topology}{%
\subsection{Retrieval pathway topology}\label{retrieval-pathway-topology}}

As important step, pathway topology information needs to be retrieved from a chosen database.
Using KEGG as an example, the retrieved topology information will be stored as a list where each element corresponds to a pathway and the numbers in the matrices encode gene-gene interaction.

\begin{Shaded}
\begin{Highlighting}[]
\NormalTok{gsTopology }\OtherTok{\textless{}{-}} \FunctionTok{retrieve\_topology}\NormalTok{(}\AttributeTok{database =} \StringTok{"kegg"}\NormalTok{, }\AttributeTok{species =} \StringTok{"hsapiens"}\NormalTok{)}
\end{Highlighting}
\end{Shaded}

Instead of downloading the topology matrices of all pathways, it is also possible to provide a restricted set of keywords for a targeted analysis.
Customised weights could be assigned to different types of gene-gene interaction type by providing a named numeric vector.

\begin{Shaded}
\begin{Highlighting}[]
\CommentTok{\# Only retrieve the topology matrices of metabolism{-} or signalling{-}related pathways}
\NormalTok{gsTopology\_sub }\OtherTok{\textless{}{-}} \FunctionTok{retrieve\_topology}\NormalTok{(}\AttributeTok{database =} \StringTok{"kegg"}\NormalTok{,}\AttributeTok{species =} \StringTok{"hsapien"}\NormalTok{, }
                                    \AttributeTok{keyword =} \FunctionTok{c}\NormalTok{(}\StringTok{"metabolism"}\NormalTok{, }\StringTok{"estrogen"}\NormalTok{))}
\FunctionTok{names}\NormalTok{(gsTopology\_sub)}
\end{Highlighting}
\end{Shaded}

In addition to focusing on one database, users could provide a vector to the \texttt{database} parameter to retrieve topology information from multiple database

\begin{Shaded}
\begin{Highlighting}[]
\NormalTok{gsTopology\_2databases }\OtherTok{\textless{}{-}} \FunctionTok{retrieve\_topology}\NormalTok{(}\AttributeTok{database =} \FunctionTok{c}\NormalTok{(}\StringTok{"kegg"}\NormalTok{, }\StringTok{"reactome"}\NormalTok{), }
                                           \AttributeTok{species =} \StringTok{"hsapien"}\NormalTok{)}
\end{Highlighting}
\end{Shaded}

\hypertarget{score-single-sample-pathway-perturbation}{%
\subsection{Score single-sample pathway perturbation}\label{score-single-sample-pathway-perturbation}}

To compute the single-sample expression changes (logFC) needed for perturbation scores, each treated sample must have a matching control sample, with the factor defining the paired structure passed to the \texttt{weight\_ss\_fc()} function through the \texttt{groupBy} parameter.
In our example dataset, pre- and post-treatment samples are matched by patient IDs.
Additionally, the sample metadata must include the treatment of all samples. The treatment column must be a factor with the reference level set to be the control treatment.

\begin{Shaded}
\begin{Highlighting}[]
\CommentTok{\# Change the treatment column to factors with treatment{-}naive as the reference}
\NormalTok{sample\_meta }\OtherTok{\textless{}{-}} \FunctionTok{mutate}\NormalTok{(}
\NormalTok{    sample\_meta, }
    \AttributeTok{treatment =} \FunctionTok{factor}\NormalTok{(}
\NormalTok{        treatment, }\AttributeTok{levels =} \FunctionTok{c}\NormalTok{(}\StringTok{"treatment{-}naive"}\NormalTok{, }\StringTok{"post{-}NACT"}\NormalTok{))}
\NormalTok{)}
\NormalTok{weightedFC }\OtherTok{\textless{}{-}} \FunctionTok{weight\_ss\_fc}\NormalTok{(}\FunctionTok{as.matrix}\NormalTok{(logCPM), }\AttributeTok{metadata =}\NormalTok{ sample\_meta,}
 \AttributeTok{sampleColumn =} \StringTok{"sample"}\NormalTok{, }\AttributeTok{groupBy =} \StringTok{"patient\_id"}\NormalTok{, }\AttributeTok{treatColumn =} \StringTok{"treatment"}\NormalTok{)}
\FunctionTok{glimpse}\NormalTok{(weightedFC)}
\end{Highlighting}
\end{Shaded}

\begin{verbatim}
## List of 2
##  $ weight: num [1:10098] 9.38e-05 1.19e-04 7.63e-05 1.18e-04 1.21e-04 ...
##  $ logFC : num [1:10098, 1:11] -1.52e-05 -2.39e-05 2.00e-04 -1.53e-04 3.84e-05 ...
##   ..- attr(*, "dimnames")=List of 2
##   .. ..$ : chr [1:10098] "ENTREZID:643837" "ENTREZID:26155" "ENTREZID:84069" "ENTREZID:57801" ...
##   .. ..$ : chr [1:11] "EOC372_post-NACT" "EOC443_post-NACT" "EOC540_post-NACT" "EOC3_post-NACT" ...
\end{verbatim}

The output of \texttt{weight\_ss\_fc()} is a \texttt{list} where one element is a matrix of weighted single-sample logFCs (\(\delta_{ghi}^*\)), with rows corresponding to genes and columns to samples, and the other element is the vector of gene-wise weights (\(w_g\)) used to calculate the weighted logFC (\(\delta_{ghi}^*\)), as described above.

The matrix of \(\delta_{ghi}^*\) values are then passed to pathway topologies to compute the gene-wise perturbation scores for all genes included in a pathway, before being summed into a single score for each pathway.
Character string \texttt{ENTREZID} was added to all row names of the \(\delta_{ghi}^*\) matrix to align with the format that pathway topologies were retrieved in.
These gene-wise perturbation scores can also be used in downstream analysis to identify genes playing the most significant roles in each pathway, as will be demonstrated in the visualisation section below.
The pathway-level perturbation scores (\(S_{hip}\)) are returned as a data.frame containing sample and gene-set names.
In the following steps, we first calculate the gene-level contributions to each pathway (\(S_{ghip}\)) using the function \texttt{raw\_gene\_pert()} and then obtain pathway-level summaries using \texttt{pathway\_pert()}.
Pathways with zero perturbatio scorse across all genes and samples will be dropped at this stage.

\begin{Shaded}
\begin{Highlighting}[]
\NormalTok{genePertScore }\OtherTok{\textless{}{-}} \FunctionTok{raw\_gene\_pert}\NormalTok{(weightedFC}\SpecialCharTok{$}\NormalTok{logFC, gsTopology)}
\NormalTok{ssPertScore }\OtherTok{\textless{}{-}} \FunctionTok{pathway\_pert}\NormalTok{(genePertScore)}
\FunctionTok{head}\NormalTok{(ssPertScore)}
\end{Highlighting}
\end{Shaded}

\begin{verbatim}
##             sample        score                                        gs_name
## 1 EOC372_post-NACT  0.005134216 kegg.EGFR tyrosine kinase inhibitor resistance
## 2 EOC443_post-NACT -0.001959631 kegg.EGFR tyrosine kinase inhibitor resistance
## 3 EOC540_post-NACT -0.006543325 kegg.EGFR tyrosine kinase inhibitor resistance
## 4   EOC3_post-NACT -0.003840943 kegg.EGFR tyrosine kinase inhibitor resistance
## 5  EOC87_post-NACT -0.003917201 kegg.EGFR tyrosine kinase inhibitor resistance
## 6 EOC136_post-NACT -0.008308897 kegg.EGFR tyrosine kinase inhibitor resistance
\end{verbatim}

\hypertarget{sample-permutation-for-normalisation-and-significance-testing}{%
\subsection{Sample permutation for normalisation and significance testing}\label{sample-permutation-for-normalisation-and-significance-testing}}

The values obtained from each pathway will vary greatly due to the variability in topologies.
To determine the significance of individual scores and transform scores to ensure they are comparable across pathways, sSNAPPY utilises a sample-permutation strategy to simulate the null distributions of perturbation scores.
Since sample labels will be permuted randomly, sample metadata is not required by the generate\_permuted \_scores function, instead, users only need to specify the number of treatment groups in the study, including the control level.
Since permutation requires a large amount of computational time and memory, sSNAPPY utilises the BiocParallel backend\citep{BiocParallel}, also allowing for customisation by the user.
Paralleled with 8 cores, permutating 1000 times took approximately 30 minutes to complete on a local laptop.
Whilst permutations can be performed an a local machine, this strategy also allows for performing the permutation steps on a HPC cluster or similar.

\begin{Shaded}
\begin{Highlighting}[]
\FunctionTok{set.seed}\NormalTok{(}\DecValTok{123}\NormalTok{)}
\NormalTok{permutedScore }\OtherTok{\textless{}{-}} \FunctionTok{generate\_permuted\_scores}\NormalTok{(}
    \FunctionTok{as.matrix}\NormalTok{(logCPM), }\AttributeTok{numOfTreat =} \DecValTok{2}\NormalTok{, }\AttributeTok{NB =} \DecValTok{1000}\NormalTok{, }
    \AttributeTok{gsTopology =}\NormalTok{ gsTopology, }\AttributeTok{weight =}\NormalTok{ weightedFC}\SpecialCharTok{$}\NormalTok{weight}
\NormalTok{)}
\end{Highlighting}
\end{Shaded}

Apart from pathways whose permuted perturbation scores are all zeros, the rest of the pathways' empirical distributions should be approximately normally distributed with means equal to zero.
To demonstrate that, we randomly sampled 6 pathways and visaulised theIR permuted perturbation scores as boxplots(Figure \ref{fig:Figure1}).

\begin{Shaded}
\begin{Highlighting}[]
\FunctionTok{set.seed}\NormalTok{(}\DecValTok{234}\NormalTok{)}
\NormalTok{permutedScore }\SpecialCharTok{\%\textgreater{}\%}
    \FunctionTok{keep}\NormalTok{(}\SpecialCharTok{\textasciitilde{}}\FunctionTok{all}\NormalTok{(.}\SpecialCharTok{!=}\DecValTok{0}\NormalTok{)) }\SpecialCharTok{\%\textgreater{}\%}
\NormalTok{    .[}\FunctionTok{sample}\NormalTok{(}\FunctionTok{seq\_along}\NormalTok{(.), }\DecValTok{6}\NormalTok{)] }\SpecialCharTok{\%\textgreater{}\%}
    \FunctionTok{as.data.frame}\NormalTok{() }\SpecialCharTok{\%\textgreater{}\%}
    \FunctionTok{pivot\_longer}\NormalTok{(}
        \AttributeTok{cols =} \FunctionTok{everything}\NormalTok{(), }
        \AttributeTok{names\_to =} \StringTok{"gs\_name"}\NormalTok{, }
        \AttributeTok{values\_to =} \StringTok{"score"}
\NormalTok{    ) }\SpecialCharTok{\%\textgreater{}\%}
    \FunctionTok{mutate}\NormalTok{(}
        \AttributeTok{gs\_name =} \FunctionTok{str\_replace\_all}\NormalTok{(gs\_name, }\StringTok{"}\SpecialCharTok{\textbackslash{}\textbackslash{}}\StringTok{."}\NormalTok{, }\StringTok{" "}\NormalTok{)}
\NormalTok{    ) }\SpecialCharTok{\%\textgreater{}\%}
    \FunctionTok{ggplot}\NormalTok{(}
        \FunctionTok{aes}\NormalTok{(gs\_name, score, }\AttributeTok{fill =}\NormalTok{ gs\_name)}
\NormalTok{    ) }\SpecialCharTok{+}
    \FunctionTok{geom\_boxplot}\NormalTok{() }\SpecialCharTok{+}
    \FunctionTok{scale\_x\_discrete}\NormalTok{(}\AttributeTok{labels =} \ControlFlowTok{function}\NormalTok{(x) }\FunctionTok{str\_wrap}\NormalTok{(x, }\AttributeTok{width =} \DecValTok{10}\NormalTok{)) }\SpecialCharTok{+}
    \FunctionTok{scale\_fill\_discrete}\NormalTok{(}\AttributeTok{name =} \StringTok{"Gene{-}set Name"}\NormalTok{) }\SpecialCharTok{+} 
    \FunctionTok{labs}\NormalTok{(}
        \AttributeTok{x =} \StringTok{"Gene{-}set Name"}\NormalTok{, }
        \AttributeTok{y =} \StringTok{"Permuted Perturbation Score"}\NormalTok{) }\SpecialCharTok{+}
    \FunctionTok{theme\_bw}\NormalTok{(}
\NormalTok{    )}
\end{Highlighting}
\end{Shaded}

\begin{figure}

{\centering \includegraphics[width=1\linewidth]{sSNAPPY_paper_files/figure-latex/Figure1-1} 

}

\caption{Permuted perturbation scores of six randomly selected pathways. All sampled empirical distributions are approximately normally distributed with a mean of zero.}\label{fig:Figure1}
\end{figure}

Permuted distributions are then used to convert each pathway-level score into a scaled robust \(Z\)-score using the function normalise\_by\_permu.
Robust Z-scores can in turn be transformed into two-sided p-values and corrected for multiple testing using any of the available methods, and returning the FDR adjusted values by default.
Users can choose to sort the output by p-values, gene-set names or sample names through the \texttt{sortBy} parameter.
In our example data, no pathways would be considered as significantly perturbed at the single-sample level using an FDR adjustment with \(\alpha\) = 0.05.

\begin{Shaded}
\begin{Highlighting}[]
\NormalTok{normalisedScores }\OtherTok{\textless{}{-}} \FunctionTok{normalise\_by\_permu}\NormalTok{(permutedScore, ssPertScore, }
                                       \AttributeTok{sortBy =} \StringTok{"pvalue"}\NormalTok{)}
\FunctionTok{head}\NormalTok{(normalisedScores)}
\end{Highlighting}
\end{Shaded}

\begin{verbatim}
##               MAD        MEDIAN                                       gs_name
## 388  0.0275353965  1.538283e-04             kegg.Epstein-Barr virus infection
## 417  0.0002129193  7.921798e-07               kegg.Autoimmune thyroid disease
## 421  0.0002129193  7.921798e-07                      kegg.Allograft rejection
## 1142 0.0034661258 -1.673836e-06 kegg.C-type lectin receptor signaling pathway
## 505  0.0098582420 -1.139206e-04                kegg.Th17 cell differentiation
## 1530 0.0036864141  2.507001e-05                       kegg.Autophagy - animal
##                sample         score   robustZ      pvalue adjPvalue
## 388  EOC136_post-NACT -0.0817379496 -2.974055 0.002938926 0.2392677
## 417  EOC136_post-NACT -0.0006237555 -2.933261 0.003354221 0.2392677
## 421  EOC136_post-NACT -0.0006237555 -2.933261 0.003354221 0.2392677
## 1142 EOC349_post-NACT  0.0085037555  2.453872 0.014132724 0.9803990
## 505  EOC153_post-NACT  0.0238610587  2.431973 0.015016823 0.7618825
## 1530 EOC443_post-NACT  0.0087762111  2.373890 0.017601798 1.0000000
\end{verbatim}

A key question of interest in this dataset is to identify which biological processes were impacted by chemotherapy across the entire group of patients.
Using the sample-level output obtained above, we can explore this by applying t-tests or regression models across all samples.
In order to minimise spurious results, Smyth's moderated t-statistics\citep{Smyth_2004} are able to be applied across the complete dataset, however given that we have robust Z-score, a constant variance should be assumed across all pathways.
To perform this analysis, we convert the robust Z-scores to a matrix then use the standard limma methodologies.
For this use case, no design matrix is required and a simple t-test is appropriate.

\begin{Shaded}
\begin{Highlighting}[]
\NormalTok{z\_matrix }\OtherTok{\textless{}{-}}\NormalTok{ normalisedScores }\SpecialCharTok{\%\textgreater{}\%}
\NormalTok{  dplyr}\SpecialCharTok{::}\FunctionTok{select}\NormalTok{(robustZ, gs\_name, sample) }\SpecialCharTok{\%\textgreater{}\%}
  \FunctionTok{pivot\_wider}\NormalTok{(}
    \AttributeTok{names\_from =} \StringTok{"sample"}\NormalTok{, }\AttributeTok{values\_from =} \StringTok{"robustZ"}
\NormalTok{  ) }\SpecialCharTok{\%\textgreater{}\%}
  \FunctionTok{column\_to\_rownames}\NormalTok{(}\StringTok{"gs\_name"}\NormalTok{) }\SpecialCharTok{\%\textgreater{}\%}
  \FunctionTok{as.matrix}\NormalTok{()}
\NormalTok{z\_fits }\OtherTok{\textless{}{-}} \FunctionTok{lmFit}\NormalTok{(z\_matrix, }\AttributeTok{design =} \FunctionTok{rep}\NormalTok{(}\DecValTok{1}\NormalTok{, }\FunctionTok{ncol}\NormalTok{(z\_matrix))) }\SpecialCharTok{\%\textgreater{}\%} 
  \FunctionTok{eBayes}\NormalTok{(}\AttributeTok{trend =} \ConstantTok{FALSE}\NormalTok{)}
\end{Highlighting}
\end{Shaded}

Pathways with an FDR \textless{} 0.05 in the moderated t-test were considered to be significantly perturbed at the group level. 22 out of the 315 tested KEGG pathways passed the selection threshold (Table \ref{tab:Table1}).

\begin{Shaded}
\begin{Highlighting}[]
\NormalTok{table1 }\OtherTok{\textless{}{-}} \FunctionTok{topTable}\NormalTok{(z\_fits, }\AttributeTok{number =} \ConstantTok{Inf}\NormalTok{) }\SpecialCharTok{\%\textgreater{}\%}
  \FunctionTok{rownames\_to\_column}\NormalTok{(}\StringTok{"gs\_name"}\NormalTok{) }\SpecialCharTok{\%\textgreater{}\%}
\NormalTok{  dplyr}\SpecialCharTok{::}\FunctionTok{filter}\NormalTok{(adj.P.Val }\SpecialCharTok{\textless{}} \FloatTok{0.05}\NormalTok{) }\SpecialCharTok{\%\textgreater{}\%} 
  \FunctionTok{mutate}\NormalTok{(}
    \AttributeTok{Direction =} \FunctionTok{ifelse}\NormalTok{(logFC }\SpecialCharTok{\textless{}} \DecValTok{0}\NormalTok{, }\StringTok{"Inhibited"}\NormalTok{, }\StringTok{"Activated"}\NormalTok{), }
    \AttributeTok{gs\_name =} \FunctionTok{str\_remove\_all}\NormalTok{(gs\_name, }\StringTok{"kegg."}\NormalTok{)}
\NormalTok{  ) }\SpecialCharTok{\%\textgreater{}\%}
\NormalTok{  dplyr}\SpecialCharTok{::}\FunctionTok{select}\NormalTok{(}
    \AttributeTok{Pathway =}\NormalTok{ gs\_name, }\AttributeTok{Change =}\NormalTok{ logFC,  P.Value, }\AttributeTok{FDR =}\NormalTok{ adj.P.Val, Direction}
\NormalTok{  )}
\end{Highlighting}
\end{Shaded}

\begin{table}

\caption{\label{tab:Table1}Significantly impacted KEGG pathways identified among post-chemotherapy samples using sSNAPPY}
\centering
\resizebox{\linewidth}{!}{
\begin{tabular}[t]{lrrrl}
\toprule
Pathway & Change & P.Value & FDR & Direction\\
\midrule
\cellcolor{gray!6}{AMPK signaling pathway} & \cellcolor{gray!6}{0.88} & \cellcolor{gray!6}{1.2e-05} & \cellcolor{gray!6}{0.0026} & \cellcolor{gray!6}{Activated}\\
p53 signaling pathway & 0.88 & 2.4e-04 & 0.0190 & Activated\\
\cellcolor{gray!6}{Progesterone-mediated oocyte maturation} & \cellcolor{gray!6}{-0.77} & \cellcolor{gray!6}{2.7e-04} & \cellcolor{gray!6}{0.0190} & \cellcolor{gray!6}{Inhibited}\\
Pathogenic Escherichia coli infection & 0.66 & 5.7e-04 & 0.0303 & Activated\\
\cellcolor{gray!6}{Cellular senescence} & \cellcolor{gray!6}{0.68} & \cellcolor{gray!6}{1.1e-03} & \cellcolor{gray!6}{0.0464} & \cellcolor{gray!6}{Activated}\\
\addlinespace
Parathyroid hormone synthesis, secretion and action & 0.63 & 1.4e-03 & 0.0464 & Activated\\
\cellcolor{gray!6}{Salmonella infection} & \cellcolor{gray!6}{0.59} & \cellcolor{gray!6}{2.2e-03} & \cellcolor{gray!6}{0.0464} & \cellcolor{gray!6}{Activated}\\
Human T-cell leukemia virus 1 infection & 0.62 & 2.3e-03 & 0.0464 & Activated\\
\cellcolor{gray!6}{Pertussis} & \cellcolor{gray!6}{0.59} & \cellcolor{gray!6}{2.3e-03} & \cellcolor{gray!6}{0.0464} & \cellcolor{gray!6}{Activated}\\
Yersinia infection & 0.62 & 2.3e-03 & 0.0464 & Activated\\
\addlinespace
\cellcolor{gray!6}{Toxoplasmosis} & \cellcolor{gray!6}{0.63} & \cellcolor{gray!6}{2.4e-03} & \cellcolor{gray!6}{0.0464} & \cellcolor{gray!6}{Activated}\\
\bottomrule
\end{tabular}}
\end{table}

For enrichment analysis in the original study\citep{Zhang2022}, unsupervised clustering was performed on all cells labelled to be cancer cells.
Identified cancer clusters were labelled by performing pathway enrichment testing on cluster marker genes.
Two clusters, associated with proliferative DNA repair signatures and stress-related markers, were found to contain significantly higher numbers of post-chemotherapy cells then pre-treatment ones (Ta in \citep{Zhang2022}).
The representative pathways enriched in the stress-associated cluster were \emph{IL6-mediated signaling events}, \emph{TNF signaling pathway}, and \emph{cellular responses to stress}, characterized by marker genes JUN, FOS, IL6, TNF, CXCR4, SNAI1, VIM, GADD45B, and MCL1.
Among the stress-related marker genes reported, 6 of them(CXCR4, FOS, GADD45B, IL6, JUN, and TNF) were implicated in pathways that were considered to be significantly impacted by sSNAPPY.
The other post-chemotherapy cell dominated cluster in the original study was enriched for pathways associated with cell proliferation and DNA repair, such as the Cell cycle, DNA repair, Homology directed repair (HDR) through homologous recombination, and Fanconi anemia pathway.
Key genes involved in those gene-sets were PCNA, CHEK1, HMGB2, BRCA2, FANCI, and POLD1.\\

Apart from treating all treated samples as biological replicates, users may wish to perform an analysis samples incorporating other phenotypic traits which may affect patients' responses to chemotherapy, such as the stages of their cancers.
To do that through the moderated t-statistic strategy and extend the above analysos, we simply need to provide an appropriate design matrix in the \texttt{lmFit} step.

\begin{Shaded}
\begin{Highlighting}[]
\NormalTok{X }\OtherTok{\textless{}{-}} \FunctionTok{model.matrix}\NormalTok{(}
    \SpecialCharTok{\textasciitilde{}}\DecValTok{0} \SpecialCharTok{+}\NormalTok{ Stage, }
    \AttributeTok{data =}\NormalTok{ sample\_meta }\SpecialCharTok{\%\textgreater{}\%}
\NormalTok{         dplyr}\SpecialCharTok{::}\FunctionTok{filter}\NormalTok{(treatment }\SpecialCharTok{==} \StringTok{"post{-}NACT"}\NormalTok{) }
    \CommentTok{\#\%\textgreater{}\%}
    \CommentTok{\#     .[match(colnames(z\_matrix), .$patient\_id),] \%\textgreater{}\%}
    \CommentTok{\#     mutate(}
    \CommentTok{\#         Stage = ifelse(Stage == "IVA", "IVA", "IIIC/IVB")}
    \CommentTok{\#     ) }
\NormalTok{    )  }\SpecialCharTok{\%\textgreater{}\%}
    \FunctionTok{set\_colnames}\NormalTok{(}\FunctionTok{str\_remove\_all}\NormalTok{(}\FunctionTok{colnames}\NormalTok{(.), }\StringTok{"Stage"}\NormalTok{)) }\SpecialCharTok{\%\textgreater{}\%}
\NormalTok{    .[,}\FunctionTok{colSums}\NormalTok{(.) }\SpecialCharTok{!=} \DecValTok{0}\NormalTok{]  }
\NormalTok{z\_fits2 }\OtherTok{\textless{}{-}} \FunctionTok{lmFit}\NormalTok{(z\_matrix, }\AttributeTok{design =}\NormalTok{ X) }\SpecialCharTok{\%\textgreater{}\%} 
  \FunctionTok{eBayes}\NormalTok{(}\AttributeTok{trend =} \ConstantTok{FALSE}\NormalTok{)}
\end{Highlighting}
\end{Shaded}

\hypertarget{visualise-perturbed-pathways-as-networks}{%
\subsection{Visualise perturbed pathways as networks}\label{visualise-perturbed-pathways-as-networks}}

sSNAPPY provides various visualisation functions to assist in the interpretation of results. Since biological pathways are not independent of each other and often contain redundant genes, visualising pathway analysis results as a network is a powerful way to not only intuitively summarise the results but also to facilitate the interpretations of the underlying biology. The \texttt{plot\_gs\_network()} function allows users to easily convert a list of significantly perturbed biological pathways to a network where edges between pathway nodes represent overlapping genes. Defined by the colorBy parameter, pathway nodes can be coloured by either the pathways' predicted direction of changes or the significance levels (Figure \ref{fig:Figure2}). The returned plot is a ggplot2 \citep{Wickham2009} object, meaning that components of its theme could be customized just as any other ggplots using the \texttt{ggplot2::theme} function.

\begin{Shaded}
\begin{Highlighting}[]
\CommentTok{\# Extract significantly perturbed pathway. To color the pathways}
\CommentTok{\# by perturbation status, add a column to the data.frame}
\NormalTok{sigPathway }\OtherTok{\textless{}{-}} \FunctionTok{topTable}\NormalTok{(z\_fits, }\AttributeTok{number =} \ConstantTok{Inf}\NormalTok{) }\SpecialCharTok{\%\textgreater{}\%} 
\NormalTok{    dplyr}\SpecialCharTok{::}\FunctionTok{filter}\NormalTok{(adj.P.Val }\SpecialCharTok{\textless{}} \FloatTok{0.05}\NormalTok{) }\SpecialCharTok{\%\textgreater{}\%}
    \FunctionTok{rownames\_to\_column}\NormalTok{(}\StringTok{"gs\_name"}\NormalTok{) }\SpecialCharTok{\%\textgreater{}\%}
    \FunctionTok{mutate}\NormalTok{(}
        \AttributeTok{Status =} \FunctionTok{ifelse}\NormalTok{(}
\NormalTok{            AveExpr }\SpecialCharTok{\textless{}} \DecValTok{0}\NormalTok{, }\StringTok{"Inhibited"}\NormalTok{, }\StringTok{"Activated"}\NormalTok{))}
\CommentTok{\# Plot the network structure}
\NormalTok{p1 }\OtherTok{\textless{}{-}}\NormalTok{ sSNAPPY}\SpecialCharTok{::}\FunctionTok{plot\_gs\_network}\NormalTok{(}
    \AttributeTok{normalisedScores =}\NormalTok{ sigPathway ,}
    \AttributeTok{gsTopology =}\NormalTok{ gsTopology, }
    \AttributeTok{colorBy =} \StringTok{"Status"}
\NormalTok{) }\SpecialCharTok{+}
    \FunctionTok{theme}\NormalTok{(}
        \AttributeTok{panel.background =} \FunctionTok{element\_blank}\NormalTok{()}
\NormalTok{    )}
\NormalTok{p2 }\OtherTok{\textless{}{-}}\NormalTok{ sSNAPPY}\SpecialCharTok{::}\FunctionTok{plot\_gs\_network}\NormalTok{(}
    \AttributeTok{normalisedScores =}\NormalTok{ sigPathway,}
    \AttributeTok{gsTopology =}\NormalTok{ gsTopology, }
    \CommentTok{\# or color nodes by significance levels}
    \AttributeTok{colorBy  =} \StringTok{"P.Value"}
\NormalTok{) }\SpecialCharTok{+}
    \FunctionTok{theme}\NormalTok{(}
        \AttributeTok{panel.background =} \FunctionTok{element\_blank}\NormalTok{()}
\NormalTok{    )}
\FunctionTok{plot\_grid}\NormalTok{(}
\NormalTok{    p1, p2, }
    \AttributeTok{labels =} \FunctionTok{c}\NormalTok{(}\StringTok{"A"}\NormalTok{, }\StringTok{"B"}\NormalTok{)}
\NormalTok{)}
\end{Highlighting}
\end{Shaded}

\begin{figure}

{\centering \includegraphics[width=1\linewidth]{sSNAPPY_paper_files/figure-latex/Figure2-1} 

}

\caption{Significantly perturbed KEGG pathways identified among post-chemotherapy samples using sSNAPPY, colored by (A) pathways’ predicted directions of changes and (B) pathways’ -log10(p-values). Pathways with a FDR < 0.05 in the moderated t-test were included.}\label{fig:Figure2}
\end{figure}

By examining the network structure, we can see that many of the highly connected pathways playing a central role in the network are immune-related. To confirm it and further condense the information, we can summarise the network structures into key biological groups by performing community detection. Widely used in network analysis, community detection is a technique used to identify groups of nodes that are more densely connected than to any other nodes in the network\citep{Newman2004}. sSNAPPY's \texttt{plot\_community()} function is a one-stop shop for applying a community detection algorithm of the user's choice to the network structure and annotating identified communities by the most common pathway category, denoting the main biological processes perturbed in that community. Retrieved directly from the \href{https://www.genome.jp/kegg/pathway.html}{KEGG website}, we have curated the most recent categorisations of KEGG pathways and included it as part of the sSNAPPY package. Annotation of KEGG pathway communities will be automatically completed by calling the in-built database, while analyses involving other pathway databases require user-provided pathway categorisations.

The defaulted Louvain method was applied to the network of significantly perturbed biological pathways and revealed two community structures, where one was annotated to be endocrine system related and the other one was infectious disease-related (Figure \ref{fig:Figure3}).

\begin{Shaded}
\begin{Highlighting}[]
\NormalTok{sSNAPPY}\SpecialCharTok{::}\FunctionTok{plot\_community}\NormalTok{(}
    \AttributeTok{normalisedScores =}\NormalTok{ sigPathway, }
    \AttributeTok{gsTopology =}\NormalTok{ gsTopology, }
    \AttributeTok{colorBy =} \StringTok{"Status"}\NormalTok{, }
    \AttributeTok{communityMethod =} \StringTok{"louvain"}\NormalTok{, }
    \AttributeTok{lb\_size =} \DecValTok{6}\NormalTok{, }
    \AttributeTok{label.fontsize =} \DecValTok{18}\NormalTok{,}
    \AttributeTok{label.fill =} \ConstantTok{NULL}\NormalTok{,}
    \AttributeTok{label.colour =} \StringTok{"blue"}\NormalTok{,}
    \AttributeTok{con.colour =} \StringTok{"blue"}\NormalTok{,}
    \AttributeTok{label.buffer =} \FunctionTok{unit}\NormalTok{(}\DecValTok{13}\NormalTok{, }\StringTok{"mm"}\NormalTok{)}
\NormalTok{) }\SpecialCharTok{+}
    \FunctionTok{theme}\NormalTok{(}
        \AttributeTok{panel.background =} \FunctionTok{element\_blank}\NormalTok{()}
\NormalTok{    ) }\SpecialCharTok{+}
\NormalTok{    ggplot2}\SpecialCharTok{::}\FunctionTok{scale\_fill\_ordinal}\NormalTok{()}
\end{Highlighting}
\end{Shaded}

\begin{figure}

{\centering \includegraphics[width=0.8\linewidth]{sSNAPPY_paper_files/figure-latex/Figure3-1} 

}

\caption{Significantly perturbed KEGG pathways identified among post-chemotherapy samples using sSNAPPY, colored by community structures detected through the louvain algorithm. The two main biological processes perturbed by the chemo-therapy were endocrine- and immune-related.}\label{fig:Figure3}
\end{figure}

Inferred directly from the expression matrix, a key advantage of sSNAPPY is that it does not require the prior definition of differentially expressed genes, which is not always detectable in clinical datasets. However, knowing the genes that are implicated in perturbing biological pathways, particularly those that affect multiple gene-sets, can provide valuable insights for biologists seeking to formulate hypotheses about underlying biological mechanisms. Therefore, sSNAPPY presents another visualisation feature called \texttt{plot\_gs2gene}, which enables the inclusion of pathway genes in network structures. Although defaulted to colour all gene nodes in grey, users can provide a vector of logFCs to colour the genes by their changes in expression. As pathways are often made of hundreds of genes, it is recommended not to plot all genes included in perturbed pathways but filter for genes more likely to be playing a significant role, achieved through only providing logFCs of genes to plot. In this example dataset, we chose to only include genes with the top 500 magnitudes of mean logFCs (Figure \ref{fig:Figure4}).

\begin{Shaded}
\begin{Highlighting}[]
\CommentTok{\# Calculate gene{-}wise mean logFCs}
\NormalTok{meanFC }\OtherTok{\textless{}{-}} \FunctionTok{apply}\NormalTok{(weightedFC}\SpecialCharTok{$}\NormalTok{logFC, }\DecValTok{1}\NormalTok{, mean )}
\CommentTok{\# Extract the top 500 meanFCs}
\NormalTok{top500 }\OtherTok{\textless{}{-}} \FunctionTok{rank}\NormalTok{(}\DecValTok{1}\SpecialCharTok{/}\FunctionTok{abs}\NormalTok{(meanFC)) }\SpecialCharTok{\textless{}=} \DecValTok{500}
\NormalTok{fcByDir }\OtherTok{\textless{}{-}} \FunctionTok{ifelse}\NormalTok{(meanFC[top500] }\SpecialCharTok{\textgreater{}} \DecValTok{0}\NormalTok{, }\StringTok{"Up{-}Regulated"}\NormalTok{, }\StringTok{"Down{-}Regulated"}\NormalTok{)}
\end{Highlighting}
\end{Shaded}

Since pathway topologies were retrieved in Entrez IDs, by default, genes' Entrez IDs will be used to annotate gene nodes in the plot. However, users can provide a data.frame mapping Entrez IDs to their chosen identifiers through the mapRownameTo parameter. A data.frame converting Entrez IDs to ensemble gene names has been made available as part of the package.

\begin{Shaded}
\begin{Highlighting}[]
\CommentTok{\# Read in built{-}in data.frame entrez2name that matches genes’ }
\CommentTok{\# Entrez IDs to gene names}
\FunctionTok{load}\NormalTok{(}\FunctionTok{system.file}\NormalTok{(}\StringTok{"extdata"}\NormalTok{, }\StringTok{"entrez2name.rda"}\NormalTok{, }\AttributeTok{package =} \StringTok{"sSNAPPY"}\NormalTok{))}
\FunctionTok{head}\NormalTok{(entrez2name)}
\end{Highlighting}
\end{Shaded}

\begin{verbatim}
## # A tibble: 6 x 2
##   entrezid           mapTo                   
##   <chr>              <chr>                   
## 1 ENTREZID:84771     DDX11L1                 
## 2 ENTREZID:727856    DDX11L1/DDX11L9/DDX11L10
## 3 ENTREZID:100287102 DDX11L1                 
## 4 ENTREZID:100287596 DDX11L1/DDX11L9         
## 5 ENTREZID:102725121 DDX11L1                 
## 6 ENTREZID:653635    WASH7P
\end{verbatim}

\begin{Shaded}
\begin{Highlighting}[]
\NormalTok{entrez2name }\SpecialCharTok{\%\textgreater{}\%}
\NormalTok{    dplyr}\SpecialCharTok{::}\FunctionTok{filter}\NormalTok{(}\FunctionTok{str\_detect}\NormalTok{(mapTo, }\StringTok{"HLA{-}DQA2"}\NormalTok{))}
\end{Highlighting}
\end{Shaded}

\begin{verbatim}
## # A tibble: 2 x 2
##   entrezid      mapTo            
##   <chr>         <chr>            
## 1 ENTREZID:3117 HLA-DQA1/HLA-DQA2
## 2 ENTREZID:3118 HLA-DQA2
\end{verbatim}

\begin{Shaded}
\begin{Highlighting}[]
\FunctionTok{str\_subset}\NormalTok{(}\FunctionTok{names}\NormalTok{(fcByDir), }\StringTok{"ENTREZID:3117|ENTREZID:3118"}\NormalTok{)}
\end{Highlighting}
\end{Shaded}

\begin{verbatim}
## [1] "ENTREZID:3117"
\end{verbatim}

\begin{Shaded}
\begin{Highlighting}[]
\CommentTok{\# Plot the pathway{-}gene network for genes with top 500 }
\CommentTok{\#magnitudes of mean FCs and label gene nodes by gene names}
\NormalTok{sSNAPPY}\SpecialCharTok{::}\FunctionTok{plot\_gs2gene}\NormalTok{(}
    \AttributeTok{normalisedScores =}\NormalTok{ sigPathway , }
    \AttributeTok{gsTopology =}\NormalTok{ gsTopology, }
    \AttributeTok{colorGsBy =} \StringTok{"Status"}\NormalTok{, }
    \AttributeTok{mapEntrezID =}\NormalTok{ entrez2name, }
    \AttributeTok{geneFC =}\NormalTok{ fcByDir, }
    \AttributeTok{filterGeneBy =} \DecValTok{0}\NormalTok{,}
    \AttributeTok{GsName\_size =} \DecValTok{4}\NormalTok{,}
     \AttributeTok{geneNameSize =} \DecValTok{5}\NormalTok{, }
    \AttributeTok{gsNodeSize =} \DecValTok{4}
\NormalTok{) }\SpecialCharTok{+}
    \FunctionTok{theme}\NormalTok{(}
        \AttributeTok{panel.background =} \FunctionTok{element\_blank}\NormalTok{()}
\NormalTok{) }\SpecialCharTok{+}
  \FunctionTok{scale\_fill\_manual}\NormalTok{(}\AttributeTok{values =} \FunctionTok{c}\NormalTok{(}\StringTok{"darkred"}\NormalTok{, }\StringTok{"lightskyblue"}\NormalTok{), }\AttributeTok{name =} \StringTok{"Pathway"}\NormalTok{) }\SpecialCharTok{+}
  \FunctionTok{scale\_colour\_manual}\NormalTok{(}\AttributeTok{values =} \FunctionTok{c}\NormalTok{(}\StringTok{"blue"}\NormalTok{, }\StringTok{"red"}\NormalTok{), }\AttributeTok{name =} \StringTok{"Gene}\SpecialCharTok{\textbackslash{}n}\StringTok{Direction"}\NormalTok{)}
\end{Highlighting}
\end{Shaded}

\begin{figure}

{\centering \includegraphics[width=0.8\linewidth]{sSNAPPY_paper_files/figure-latex/Figure4-1} 

}

\caption{Significantly perturbed KEGG pathways identified among post-chemotherapy samples using sSNAPPY and asssiociated pathway gens with top 500 magnitudes of fold-changes. Both pathways and genes were colored by their directions of changes. Only genes involced in at least 2 pathways were included.}\label{fig:Figure4}
\end{figure}

\hypertarget{identify-hub-genes-contributing-to-pathway-perturbation}{%
\subsection{Identify hub genes contributing to pathway perturbation}\label{identify-hub-genes-contributing-to-pathway-perturbation}}

If we would like to further investigate a specific pathway and elucidate the key genes that contributed to its perturbation, such as the activation of the ``p53 signalling pathway,'' we can employ a heatmap to display the gene-level perturbation scores of all the genes within the pathway and annotate each column (ie. each sample) by the direction of pathway perturbation in that sample or any other sample metadata using the plot\_gene\_contribution function (Figure \ref{fig:Figure5}).

\begin{Shaded}
\begin{Highlighting}[]
\NormalTok{annotation\_df }\OtherTok{\textless{}{-}}\NormalTok{ ssPertScore }\SpecialCharTok{\%\textgreater{}\%}
\NormalTok{    dplyr}\SpecialCharTok{::}\FunctionTok{filter}\NormalTok{(gs\_name }\SpecialCharTok{==} \StringTok{"kegg.p53 signaling pathway"}\NormalTok{) }\SpecialCharTok{\%\textgreater{}\%}
    \FunctionTok{mutate}\NormalTok{(}
        \StringTok{\textasciigrave{}}\AttributeTok{pathway{-}level}\StringTok{\textasciigrave{}} \OtherTok{=} \FunctionTok{ifelse}\NormalTok{(score }\SpecialCharTok{\textgreater{}} \DecValTok{0}\NormalTok{, }\StringTok{"Activated"}\NormalTok{, }\StringTok{"Inhibited"}\NormalTok{)}
\NormalTok{    ) }\SpecialCharTok{\%\textgreater{}\%}
    \FunctionTok{left\_join}\NormalTok{(}
\NormalTok{        sample\_meta }\SpecialCharTok{\%\textgreater{}\%}
\NormalTok{            dplyr}\SpecialCharTok{::}\FunctionTok{select}\NormalTok{(sample, Stage)}
\NormalTok{    ) }\SpecialCharTok{\%\textgreater{}\%}
\NormalTok{    dplyr}\SpecialCharTok{::}\FunctionTok{select}\NormalTok{(sample, }\StringTok{\textasciigrave{}}\AttributeTok{pathway{-}level}\StringTok{\textasciigrave{}}\NormalTok{, Stage)}
\end{Highlighting}
\end{Shaded}

\begin{verbatim}
## Joining with `by = join_by(sample)`
\end{verbatim}

\begin{Shaded}
\begin{Highlighting}[]
\FunctionTok{plot\_gene\_contribution}\NormalTok{(}
    \AttributeTok{genePertMatr =}\NormalTok{ genePertScore}\SpecialCharTok{$}\StringTok{\textasciigrave{}}\AttributeTok{kegg.p53 signaling pathway}\StringTok{\textasciigrave{}}\NormalTok{, }
    \AttributeTok{annotation\_df =}\NormalTok{  annotation\_df, }
    \AttributeTok{topGene =} \DecValTok{20}\NormalTok{, }\AttributeTok{filterBy =} \StringTok{"mean"}\NormalTok{, }
    \AttributeTok{mapEntrezID =}\NormalTok{ entrez2name,}
    \AttributeTok{annotation\_colors =} \FunctionTok{list}\NormalTok{(}
        \StringTok{\textasciigrave{}}\AttributeTok{pathway{-}level}\StringTok{\textasciigrave{}} \OtherTok{=} \FunctionTok{c}\NormalTok{(}\AttributeTok{Activated =} \StringTok{"darkred"}\NormalTok{, }\AttributeTok{Inhibited =} \StringTok{"lightskyblue"}\NormalTok{))}
\NormalTok{)}
\end{Highlighting}
\end{Shaded}

\begin{figure}

{\centering \includegraphics[width=1\linewidth]{sSNAPPY_paper_files/figure-latex/Figure5-1} 

}

\caption{Gene-level perturbation scores of all genes included in the "p53 signalling pathway" in each sample where columns of samples were annotated by pathway-level perturbation and the stages of cancers. Genes ATR and ATM were identified to be the key driverd of the activation of p53 signalling pathway. }\label{fig:Figure5}
\end{figure}

From this heatmap we can easily identify that the hub genes making the biggest contribution to the activation of p53 signalling pathway upon chemotherapy were gene ART and gene ATM. The Ataxia-telangiectasia mutated (ATM) gene is a well-established oncosuppressor\citep{Moslemi2021}, mutation of which has been observed in many types of cancers\citep{Choi2016}. Also involved in DNA damage repair, the Ataxia telangiectasia and RAD3-related protein kinase (ATR) gene has been shown to be a promising therapeutic target for HGSOC\citep{Li2022}.

\hypertarget{discussion}{%
\section{Discussion}\label{discussion}}

In conclusion, the paper showcased an R/Bioconductor package that offers a novel single-sample pathway perturbation testing approach. sSNAPPY utilizes pathway topology information to compute perturbation scores that predict pathways' potential directions of changes in individual samples. This approach addresses the limitations of current strategies that fail to account for gene-gene interactions encoded by pathway topologies or predict the directionality of pathway activities. By applying sSNAPPY to a public scRNA-seq data collected before and after HGSOC patients were subjected to chemotherapy, we demonstrated its ability to detect significant pathway perturbations of various interesting biological processes beyond what were shown in the original studies. Overall, sSNAPPY presents a promising strategy for single sample-based pathway analysis in RNA-seq data.

\hypertarget{data-availability}{%
\section{Data availability}\label{data-availability}}

The dataset analysed in this manuscript are stored in the data directory of this GitHub repository.

\hypertarget{software-availability}{%
\section{Software availability}\label{software-availability}}

\begin{itemize}
\item
  Software available from: \url{https://bioconductor.org/packages/release/bioc/html/sSNAPPY.html}
\item
  Source code available from: \url{https://github.com/Wenjun-Liu/sSNAPPY}
\item
  Archived source code at time of publication: {[}DOI (found on right hand side of a Zenodo record){]}
\item
  License: \href{https://opensource.org/license/mit/}{MIT}
\end{itemize}

\hypertarget{competing-interests}{%
\section{Competing interests}\label{competing-interests}}

No competing interests were disclosed

\hypertarget{grant-information}{%
\section{Grant information}\label{grant-information}}

Any grants that supported the work must be listed here, including the grant number.

\renewcommand\refname{Acknowledgements}
{\small\bibliography{sample.bib}}

\end{document}
